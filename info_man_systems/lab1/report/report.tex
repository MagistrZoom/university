\documentclass[a4paper, 12pt]{article}
\usepackage{geometry}
\geometry{verbose,a4paper,tmargin=2cm,bmargin=2cm,lmargin=2cm,rmargin=2cm}
\usepackage{fontspec}
\setmainfont[
  Ligatures=TeX,
  Extension=.otf,
  UprightFont=*-regular,
  ItalicFont=*-italic,
  BoldFont=*-bold,
  BoldItalicFont=*-bolditalic,
]{xits}
\usepackage[english,russian]{babel}

\newif\ifisinsp
\newif\ifisone
\newif\ifisname
\newif\ifisnum
\isinsptrue
\isonetrue
\isnamefalse
\isnumtrue

\def \labtype {Лабораторная}
% Это для нумерации страниц после титульника
\usepackage{fancyhdr}
\pagestyle{fancy}
\renewcommand{\headrulewidth}{0pt}
\fancyfoot[C] {\thepage}

\isonefalse
\isnametrue
\def \labnum {1}
\def \labsubj {Информационно-управляющие системы}
\def \labauthor {Айтуганов Д. А. \\ Чебыкин И. Б.}
\def \labgroup {P3401}
\def \labinsp {Пинкевич В. Ю.}
\def \labname {Вариант 4}

\usepackage{graphicx,tabularx}

\usepackage{caption}
\usepackage{verbatim}
\usepackage[dvipsnames]{xcolor}

\usepackage{fancyvrb}

\RecustomVerbatimCommand{\VerbatimInput}{VerbatimInput} {
 fontsize=\scriptsize,
 %
 frame=lines,  % top and bottom rule only
 framesep=2em, % separation between frame and text
 rulecolor=\color{Gray},
 %
 label=\fbox{\color{Black}source},
 labelposition=topline,
 %
}

\captionsetup{labelsep=period}
\pagestyle{fancy}
\begin{document}
\begin{titlepage}
	\begin{center}
		\large
		Университет ИТМО

		\vspace{0.25cm}
		
		Факультет программной инженерии и компьютерной техники
		
		Кафедра вычислительной техники
		\vfill
		
		\textsc{\labtype\spaceработа \ifisnum № \labnum{} \fi по дисциплине \\"\labsubj" \ifisname\small \\ \labname \fi}
			
		\bigskip
	\end{center}
	\vfill
	\vfill
	
	\begin{flushright}
	\ifisone
	Выполнил: \labauthor
	\else
	Выполнили: \labauthor
	\fi

	\vspace{0.25cm}
	Группа: \labgroup
			
	\vspace{0.25cm}
	\ifisinsp
	Проверяющий: \labinsp
	\fi
	\end{flushright}
	\vfill
	
	\begin{center}
	СПб, \the\year
	\end{center}
\end{titlepage}

\section{Задание}
Разработать и реализовать драйверы светодиодных индикаторов и DIP-переключателей
контроллера SDK-1.1. Написать тестовую программу с использованием разработанных
драйверов по алгоритму, соответствующему варианту задания.

В случае установки на DIP-переключателях кода 0x44 (шестнадцатеричное значение) на
светодиодные индикаторы должна выводиться анимация, показанная ниже. Во всех
остальных случаях светодиодные индикаторы отражают инвертированное значение,
выставленное на DIP-переключателях.

\section{Блок-схема программы}
\section{Исходный код}
\VerbatimInput[label=main.c]{src/src/lab.c}
\section{Основные езультаты}

\end{document}
